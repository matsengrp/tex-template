\documentclass{article}

\usepackage{amsmath}
\usepackage{amsfonts}
\usepackage{amssymb}
\usepackage{amsthm}
\usepackage{graphicx}
\usepackage[hidelinks]{hyperref}
\usepackage[inline]{showlabels}
\renewcommand{\showlabelfont}{\tiny\sffamily}
\usepackage{lineno}
\linenumbers

\newtheorem{lemma}{Lemma}
\newtheorem{prop}{Proposition}
\newtheorem{thm}{Theorem}
\newtheorem{prob}{Problem}
\newtheorem{defn}{Definition}
\newtheorem{obs}{Observation}
\newtheorem{alg}{Algorithm}

\newcommand{\median}{\operatorname{median}}

% http://bytesizebio.net/2013/03/11/adding-supplementary-tables-and-figures-in-latex/
\newcommand{\beginsupplement}{%
        \setcounter{table}{0}
        \renewcommand{\thetable}{S\arabic{table}}%
        \setcounter{figure}{0}
        \renewcommand{\thefigure}{S\arabic{figure}}%
     }

\hyphenation{Ge-nome Ge-nomes hyper-mut-ation through-put}

\title{Template}
\author{You}

\begin{document}
\maketitle

\begin{abstract}
Abstract here.
\end{abstract}


\section*{Introduction}
Hooray for Joe~\cite{Felsenstein1981-zs}.

% \begin{figure}[h]
% \centering
% \includegraphics[width=0.35\textwidth]{figures/subsplit.pdf}
% \caption{\
% A subsplit structure.
% }%
% \label{fig:subsplit}
% \end{figure}


\bibliographystyle{plain}
\bibliography{main}


% \clearpage
% \section*{Supplementary Materials}
% \beginsupplement
% Supplementary text and figures here.


\end{document}
